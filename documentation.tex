\documentclass[12pt, letterpaper]{article}
\usepackage[utf8]{inputenc}
\usepackage{amsmath}    % Extended typesetting of mathematical expression.
\usepackage{amssymb}    % Provides a multitude of mathematical symbols.
\usepackage{mathtools}  % Further extensions of mathematical typesetting.
\usepackage{footnote}
\usepackage{url}

\DeclareMathOperator*{\minimize}{minimize}
\DeclareUnicodeCharacter{00A0}{ }
\makesavenoteenv{tabular}
\makesavenoteenv{table}

\title{Vehicle Routing Problem approaches with Answer Set Programming}
\author{André Schidler}
\begin{document}
\maketitle

\section{Introduction}
The problem of solving computationally hard problems is the focus of much research, as it is tied to many real world problems. The vehicle routing problem (VRP) is an intractable optimization problem that has been the subject of extensive research.\\
One of the problems appeals is its direct applicability to real world scenarios. Therefore a lot of approaches of solving this problem and extensions of it have been developed, both for exact and heuristically.

The goal of this project was to gauge the applicability of answer set programming (ASP) to the vehicle routing problem. ASP has been chosen because of the authors interest in this approach. The VRP was selected as a representative of the larger class of scheduling problems.

This document will introduce the exact definition if the VRP, before it discusses the solution that has been created in the course the project and concludes with the results found.

\section{Problem Description}
There exists no single definition of the VRP. While the core idea is represented in every definition, details change from source to source. Therefore an informal description of the idea behind the VRP is given first and afterwards the formal definition used in this project is given.

\subsection{Informal Description}
The problem of how to assign delivery vehicles to customers is quite common for companies. This is exactly the problem defined by the VRP. To be more exact, a solution to a VRP are routes for the vehicles, where each route must not exceed a defined time limit and every customer is serviced. Each route starts and ends at the depot.\\
While the VRP is usually an optimization problem (the routes are optimal according to some cost metric), the associated decision problem can be thought of as a generalization of the travelling salesmen problem. It is therefore $\mathcal{NP}$-hard.

Many extensions to the VRP exist. These include time windows (a node must be visited during specific times), site dependency (not every vehicle may service every node) or multiple depots.\\
In the course of the project a solver for the time window extension has been created as well.\\

As already mentioned the formal definitions of the VRP vary \cite{survey} \cite{vrptw} \cite{complete}. To give a feeling for these variations here are some examples:
\begin{itemize}
	\item Nodes may be defined via graphs, complete graphs or points in a 2-D grid.
	\item In some definitions, each node must be visited exactly once, in others it may be visited multiple times.
	\item In some definitions there is a difference between costs and duration, in others there is no distinction.
\end{itemize}

\subsection{Formal Description}
For this project, a connected and directed graph is used. Each node may be visited multiple times and there is a distinction between costs and duration. Costs are used for optimization and duration is used to limit route length.\\
A graph is used, because it fits the ASP language better than using other definition methods. Allowing for a node to be visited multiple times, allows for leaf nodes (the predecessor needs to be visited at least twice) and ''choke nodes'' where multiple vehicles must use the same node. Finally a separation between cost and duration is common in other papers and there was no reason the merge the concepts.

A VRP instance therefore consists of the following elements:
\begin{itemize}
	\item A graph $G = (C,E)$ with vertices $V$ and edges $E$.
	\item A mapping from edges to integers, called duration $d(x,y) = i, (x,y) \in E, i \in \mathbb{N}$.
	\item A mapping from edges to integers, called costs $c(x,y) = i, (x,y) \in E, i \in \mathbb{N}$.
	\item A mapping from vertices to integers, called the service time $d(x) = i, x \in V, i \in \mathbb{N}$.
	\item A time limit $L \in \mathbb{N}$.
	\item A set of vehicles $V$.
\end{itemize}

A solution to an VRP instance consists of a route for each vehicle and a mapping determining which vehicle services which node (in case of multiple visits). Formally these are defined as follows:

\begin{enumerate}
	\item A route is defined for every vehicle $v \in V$: \\
		$r_v(x,i) =
		\begin{cases}
			1	\quad \text{if vehicle } v \text{ is at node } x \in C \text{ at step } i\\
			0	\quad \text{else}
		\end{cases}
		$
	\item The service mapping is defined for each node $x \in C$: \\
		$s_x(v) =
		\begin{cases}
			1	\quad \text{if vehicle } v \in V \text{ services node } x\\
			0	\quad \text{else}
		\end{cases}
		$
\end{enumerate}

Each solution must adhere to constraints. Each route must start and end at the depot (known as the 0 node) and must not exceed the maximum duration. Formally the following constraints must hold for every solution:
\begin{itemize}
	\item Routes must start at the depot:
		$$r_v(0,1) = 1 \quad \text{for all } v \in V$$
	\item Routes must end at the depot:
		$$r_v(0,n) = 1 \quad n = |r_v| \text{, for all } v \in V$$
	\item A vehicle is at exactly one node at each step: 
		$$\sum_{x \in C} r_v(x,i) = 1 \quad 1 \leq i \leq |r_v|$$
	\item Each node is serviced by exactly one vehicle:
		$$\sum_{v \in V} s_x(v) = 1 \quad \text{for all } x \in C$$
	\item Each node transition must exist as an edge: 
		$$\sum_{x,y \in C, (x,y) \not\in E} \sum_{v \in V} \sum_{i=2} (r_v(x,i-1) * r_v(y,i)) = 0 $$
	\item A vehicle that services a node must visit it: 
		$$\sum_{i=1} r_v(x,i) \geq s_x(v) \quad \text{for all } v \in V, x \in C $$
	\item The total travel duration is within the limit: 
		$$\sum_{i=2} \sum_{(x,y) \in E} (d(x,y) * r_v(x,i-1) * r_v(y,i)) + \sum_{x \in C}(s_x(v) * s(x)) \leq L \quad \text{for each } v \in V$$
	\item The depot is not serviced: 
		$$\sum_{v \in V}s_0(v) = 0$$

\end{itemize}

Finally the VRP is an optimization problem. Each edge has an associated cost and the goal is to minimize the total cost over all routes:
$$\displaystyle{\minimize \sum_{v \in V} \sum_{i=2}\sum_{(x,y) \in E} (c(x,y) * r_v(x,i-1) * r_v(y,i))}$$

\subsection{Time Window Extension}
The time window extension to the VRP (VRPTW) adds another constraint to the original problem. Each customer only accepts deliveries within defined time windows.

Formally an instance additionally defines a set of times $W_x \subset \mathbb{N}$ for each node $x \in C \setminus \{0\}$.

Assuming S is a solution for the VRP part of the problem. In order to check if S also solves the VRPTW, we define two mappings:
\begin{enumerate}
	\item For each vehicle $v \in V$ and each step $1 \leq i \leq |r_v|$ we define the arrival time $a_v(i) \in \mathbb{N}$.
	\item For each node $x \in C \setminus {0}$ we define the service step $t_x(i)$ as:
		$$t_x(i) = \begin{cases}
			1	\quad \text{if node } x \text{ is serviced at step } i \in \mathbb{N}\\
			0	\quad \text{else}
		\end{cases}$$ 
\end{enumerate}

For S to be a solution to the VRPTW instance, there must exist such mappings that the following constraints hold:
\begin{itemize}
	\item A node is only serviced once:\\
		$$\sum{i=1} t_x(i) = 1 \quad \text{for each } x \in V \setminus {0}$$
	\item The servicing vehicle must be at the node at the service step:\\
		$$\sum_{v \in C} \sum{i=1} (s_x(v) *  t_x(i) * r_v(x,i)) = 1 \quad \text{for each } x \in V \setminus {0}$$
	\item Arrival times must be consistent with previous arrival:\\
		$$a_v(i) \geq a_v(i-1) + \sum_{x,y \in V} (d(x,y) * r_v(x,i-1) * r_v(y,i)) + \sum_{x \in V} (r_v(x, i-1) * s_x(v) * t_x(i))$$
	\item The total duration must not exceed the allowed duration: \\
		$$a_v(n) \leq L \quad n = |a_v|, \text{for each } v \in C$$
	\item The arrival times must be consistent with the time windows:\\
		$$a_v(i) \in W_x \quad r_v(x,i)  = 1 \text{, for all} v \in C, i \in \mathbb{N}$$
\end{itemize}

\section{Solution}
For this project two ASP encodings were created, one to solve VRP instances and one extension for VRPTW instances. Additionally a solution visualizer and a generator for larger problem instances were written.

This chapter gives an overview over the idea behind the encodings. For a detailed overview over the program code and the used atoms, please refer to the source files and README.

\subsection{VRP Encoding}
Over the course of the project the encoding has been revised several times. The first goal was to create an encoding that produces correct results. This first version was not able to solve non-trivial instances within an acceptable time frame.\\
The final version consists of three parts, which will subsequently be discussed:
\begin{enumerate}
	\item Route generation
	\item Constraint checking
	\item Symmetry breaking
\end{enumerate}

The job of the route generation is to produce solution candidates, that are then checked by the constraint checking part.\\
In the first version the route generation tried to construct routes for each vehicle, by using at least one outgoing edge per node. As this yields a lot of non-connected routes that may not start or end at the depot, this proved to be inefficient.\\
Another attempt to encode most of the problem with an constraint satisfaction problem extension (clingcon \citep{potassco}), yielded worse results. This may be either due to the approach not being suitable to the VRP or the non-proficiency of the author.\\
In the final version, route generation tries to find cycles, by starting at the depot and going forward, until the depot is hit again. This is continued until all nodes are visited. In a separate step these cycles are assigned to vehicles. The advantage of this approach is, that the routes generated this way adhere to almost all constraints. Only the duration limit remains to be checked.

Often two solutions are equal except for some ordering. As an example think of a cycle in an undirected graph. The two directions are two solutions, but except for the route ordering the solutions are equal. These properties are called symmetries. Eliminating such solutions reduces the amount of solutions the solver has to check. This what the last part of the encoding is for.

The encoding uses several limits that are required for ASP, to avoid infinite domains. These include the time limit, number of cycles and the maximal number of steps in a cycle. The performance of this encoding depends heavily on the tightness of these limits.

\subsection{VRPTW Encoding}
The time window encoding is a direct extension of the VRP encoding. There are two tasks the extension performs:
\begin{enumerate}
	\item Check if all the nodes can be serviced within their time windows.
	\item Check if the total travel duration still holds.
\end{enumerate}

The check of these constraints require a timeline for each route, which requires an ordering of the assigned cycles. \\
When creating the timeline it is assumed, that the vehicle does not have to leave at time 0 and it is permitted to wait at node to hit the time window.

There may be a lot of ways to order the cycles and a lot of valid timelines, depending on the size of the time windows. This makes instances of the time window extension a lot harder to solve compared to normal VRPs.

\subsection{Visualization}
In order to make the results more accessible, a little visualisation script has been added, using the clingo python interface. The script visualizes the solution in two forms:
\begin{itemize}
	\item Using the networkx \cite{networkx} library is produces a graphic of the solution. The graph is drawn, and using a color for each vehicle, each edge's color depends on the vehicles using this edge and a node is colored according to the vehicle servicing it.
	\item One line with the node sequence for each vehicle is printed to the console.
\end{itemize}


\subsection{Instance Generator}
In order to test the scripts with larger instances, a Java program which creates instances of increasing sizes has been added. Different types of instances are created, with sizes from 1 to 20, where the meaning of size depends on the instance type.

The following instance types are created by the program:
\begin{description}

\item[Connected Nodes]
This instance type produces a fully connected graph where the size determines the number of nodes (size + 2). Each instance uses five vehicles, with costs, service times and durations of 1.

\item[Connected Nodes Random]
Similar to ''Connected Nodes'' but uses random edge costs ranging from 1 to 50.

\item[Connected Vehicles]
This instance type produces a fully connected graph where the size determines the number of vehicles. Each instance has 10 nodes, with costs, service times and durations of 1.

\item[Connectedness Ratio]
This instance type produces a graph with a specific ''connectedness ratio'', i.e. how many of the possible edges are actually created (size * 5 percent). Each instance uses 5 vehicles and 20 nodes. Durations, costs and service times are 1.

\item[Connectedness Nodes]
Similar to the previous type, but uses size + 2 nodes and creates 30\% of possible edges.

\item[Cycle Number]
Produces cycles of length 10, where the size determines how many cycles are used. Size also determines how many vehicles are used. Costs, service times and durations are 1.

\item[Cycle Number Vehicle]
Similar to the previous type, but uses fixed vehicle number of 5.

\item[Cycle Size]
This type uses cycles as well, but here size determines the number of nodes per cycle. Five such cycles are created as well as five vehicles. Costs, service times and durations are 1.

\item[Simple Cycle]
This type creates one cycle with size * 5 nodes using one vehicle. Costs, service times and durations are 1.

\end{description}


\section{Results}
This section presents the results taken from the project. This includes performance measurements and possible enhancements that have been identified during the project.\\
For development and testing Potassco \cite{potassco} has been used.

\subsection{Test instances}
In order to evaluate how well the encoding works, the aforementioned test instances were run with increasing sizes. Table \ref{table:benchmark} shows the maximum size that was solvable for the VRP and VRPTW encodings in a five minute timeframe. The tests were run on an Intel Core i5 with 3,3 GHz and 88 GB of RAM.\\
To give an intuition of the performance gain that was possible between the first and last version of the encoding, the table additionally contains results taken with the first version.\\
Please note, that not all instance types exist for the time window extension. Only deterministic types are used for VRPTW. Also not all types were used with the first version, as the types were created at a later time.

\begin{table}
\begin{tabular}{l | r | r | r}
Instance Type			& VRP v1	& VRP	& VRPTW\\
\hline
Connected Nodes			& 2			& 7		& 4\\
Connected Nodes Random	& -			& 7		& -\\
Connected Vehicles		& 0			& 0 \footnote{While v1 stopped in grounding, the final encoding actually found the optimal solution, but did not manage to verify its optimality in time}	& 0\\
Connectedness Ratio		& -			& 6		& -\\
Connectedness Nodes		& -			& 18	& -\\
Cycle Number			& 2			& 5		& 2\\
Cycle Number Vehicle	& -			& 4		& -\\
Cycle Size				& 3 		& 12	& 4\\
Simple Cycle			& 7			& 9		& 3\\
\end{tabular}
\caption{Solvable test instances}
\label{table:benchmark}
\end{table}

The results show that considerable performance gain was possible due to the optimization efforts. The results also show that performance decreases drastically when solving VRPTW instances.


Unfortunately the instances available on the internet would require a greater transformation. The transformation would be time-consuming and the results would not be directly comparable with the original instance. Therefore only a rough comparison can be drawn.\\
The ''Simple Cycle'' instance is probably the easiest graph structure in the context of the VRP. The used encoding is able to solve an instance of size 9, which means 45 nodes. This is rather small compared to current benchmark results, where instances of sizes of over 100 nodes can be optimally solved in under a minute \cite{benchmark}.

\subsection{Possible Enhancements}
There are three things that are known to increase the solving time of this encoding:
\begin{itemize}
	\item Loose limits; this may be improved by making the limits part of the problem instance and therefore the users responsibility. These limits include time limit (already part of the instance), cycle length limit and the maximum number of cycles. Loose limits cause a blow up in grounding.
	\item Symmetries; these allow for a lot of equal solutions that have to be explored by the solver. Efforts have been made to break these symmetries. Symmetries that are known to exist are the assignment of cycles to vehicles, servicing assignments and route directions.
	\item Number of edges; the impact of edges are twofold, they cause a blow up in the grounding phase and allow for more solutions, increasing the solving time as well.
\end{itemize}

While it is natural for larger instances to contain more edges, the impact could be softened by introducing a preprocessing step. The extra step may eliminate unnecessary edges (e.g. in a complete graph) or identify clusters and thereby divide the problem into smaller sub-problems.

It is also worth mentioning that during this project the constraints have been loosened to be more applicable to real world scenarios. Especially allowing nodes to be visited more than once allows for a larger set of instances to be solvable, but has a huge impact on performance. Tightening this constraint has the following advantages:
\begin{itemize}
	\item A greatly reduced set of possible routes.
	\item A cycle directly corresponds to a vehicle.
	\item For each node exactly one incoming and one outgoing edge is used.
	\item Only the used edges are of interest, no order is required.
\end{itemize}

\section{Conclusion}
The conclusion is twofold. First personally I learned a lot about ASP, leaving me with a deeper understanding of how it works and how encodings can be optimized.\\
Second that ASP is probably not the optimal choice for this kind of problem. Many constraints in the VRP and especially the VRPTW require ordering and counting, which seem contrived and inefficient in ASP.

\bibliographystyle{unsrt}
\bibliography{documentation}

\end{document}